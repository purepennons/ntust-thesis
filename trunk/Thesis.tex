%
% this file is encoded in utf-8
% v1.7

\documentclass[12pt, a4paper]{ntust_report}

\input{common_env}  %基本的環境設定  無需改變  
\usepackage{amsmath}
\usepackage{cite}     % Sort all citations
\usepackage{psfrag}   % Used to insert text into figures
\usepackage{graphicx}
\usepackage{xspace}
\usepackage{color}    % Used to highlight comments
\usepackage{subfigure}
\usepackage{url}
\usepackage{cite}     % Sort all citations

\usepackage{algorithmic} % pseudocode
\usepackage{algorithm}
\usepackage{float}
\usepackage{amssymb}
\floatname{algorithm}{Method}

\usepackage{xspace}
\newcommand{\comment}{\textcolor{red}}
\newcommand{\todo}{\textcolor{blue}}
\newcommand{\jmh}[1]{\textcolor{red}{#1 -- JMH}}
\newcommand{\hc}[1]{\textcolor{blue}{#1 -- HC}}

\def\centerhack#1{\hbox to 0pt{\hss\footnotesize #1\hss}}
\def\dchack#1{\vbox to 0pt{\vss{\hbox to 0pt{\hss#1\hss}}\vss}}

\newcounter{linecounter}
\newcommand{\resetlinecounter}{\setcounter{linecounter}{1}}
\newcommand{\llabel}[1]{\arabic{linecounter}. \addtocounter{linecounter}{-1}\refstepcounter{linecounter}\label{#1}\stepcounter{linecounter}}
\newcommand{\declabel}{\addtocounter{linecounter}{-1}}

\newcommand{\aboveleftarrow}[1]{{\buildrel #1 \over \longleftarrow}}
\newcommand{\aboverightarrow}[1]{{\buildrel #1 \over \longrightarrow}}
\newcommand{\aboveleftrightarrow}[1]{{\buildrel #1 \over
    \leftrightarrow}}

\renewcommand{\paragraph}[1]{\vspace{2mm}\noindent{\textbf{#1}}\quad}

\newcommand{\sign}[2]{\textrm{Sign}_{#2}(#1)}
\newcommand{\signature}[2]{\sigma_{#2}(#1)}

\newtheorem{theorem}{Theorem}[section]
\newtheorem{conjecture}[theorem]{Conjecture}
\newtheorem{corollary}[theorem]{Corollary}
\newtheorem{proposition}[theorem]{Proposition}
\newtheorem{lemma}[theorem]{Lemma}



\begin{document}
\begin{CJK}{UTF8}{bsmi}   %%% ZZZ %%%  <<< 在這裡更改預設中文字型、編碼
% 編碼:UTF8, Bg5, ...
% 中文字型名稱:TeXLive 安裝有一套明體字 bsmi, 楷書與其他字型視你的 LaTeX CJK 系統裝設情況而定
% 注意! 修改時請一併修改此檔案末端之\end{ZZZ}
% global CJK setting
\CJKindent  %%% ZZZ %%%  段首內縮兩格

% 下列中文名詞的定義,如果以註解方式關閉取消,
% 則會以系統原先的預設值 (英文) 替代
% 名詞 \prechaptername 預設值為 Chapter
% 名詞 \postchaptername 預設值為空字串
% 名詞 \tablename 預設值為 Table
% 名詞 \figurename 預設值為 Figure
\renewcommand\prechaptername{Chapter} % 出現在每一章的開頭的「第 x 章」
\renewcommand\postchaptername{}
\renewcommand{\tablename}{Table} % 在文章中 table caption 會以「表 x」表示
\renewcommand{\figurename}{Figure} % 在文章中 figure caption 會以「圖 x」表示

% 下列中文名詞的定義,用於論文固定的各部分之命名 (出現於目錄與該頁標題)
\newcommand{\nameInnerCover}{教授推薦書}
\newcommand{\nameCommitteeForm}{論文口試委員審定書}
\newcommand{\nameCopyrightForm}{授權書}
\newcommand{\nameCabstract}{中文摘要}
\newcommand{\nameEabstract}{Abstract}
\newcommand{\nameAckn}{Acknowledgment}
\newcommand{\nameToc}{Table of contents}
\newcommand{\nameLot}{List of Tables}
\newcommand{\nameTof}{List of Figures}
%\newcommand{\nameSlist}{符號說明}
\newcommand{\nameRef}{References}
%\newcommand{\nameVita}{自傳}
 %在此檔案處定義文章中的中文名詞

	%----------------------------------------------------------------------------------------------------------------------------------------------------------
	%%% 以下是載入前頁、本文、後頁
	% 此行請勿更動
	% 如需針對個別章節獨立編譯
	% 請在 my_chapters.tex 檔裡對個別章節的 \input 指令以行首百分號方式做開關。
	%----------------------------------------------------------------------------------------------------------------------------------------------------------
	% front matter 前頁
	% 包括封面、書名頁、中文摘要、英文摘要、誌謝、目錄、表目錄、圖目錄、符號說明
	% 在撰寫各章草稿時,可以把此部份「關掉」,以節省無謂的編譯時間。
	% 實際內容由
	%    my_names.tex, my_cabstract.tex, my_eabstract.tex, my_ackn.tex, my_symbols.tex
	% 決定
	% ntust_frontpages.tex 此檔只提供整體架構的定義,不需更動
	% 在撰寫各章草稿時,可以把此部份「關掉」,以節省無謂的編譯時間。
	
	\input{frontpages/ntust_frontpages.tex} 

%----------------------------------------------------------------------------------------------------------------------------------------------------------
	% main body 論文主體。建議以「章」為檔案分割的依據。
	% 以下為建議的命名分類
	%   introduction.tex   related_work.tex  protocol.tex  evaluation.tex  conclusion.tex
	% 做為這幾個「章」的檔案名稱,並將檔案存放於資料夾 sections/ 下
	% 實際命名方式可以隨你意
	% 在撰寫各章草稿時,可以把其他章節關掉 (行首加百分號)


	\input{sections/Introduction.tex}
	\input{sections/Method.tex}
	\input{sections/Conclusion.tex}
%----------------------------------------------------------------------------------------------------------------------------------------------------------
	% back pages 後頁
	% 包括參考文獻、附錄、自傳
	% 實際內容由
	%    my_bib.bib, my_appendix.tex, my_vita.tex
	% 決定
	% ntust_backpages.tex 此檔只提供整體架構的定義,不需更動
	% 在撰寫各章草稿時,可以把此部份「關掉」,以節省無謂的編譯時間。
	\input{backpages/ntust_backpages.tex}


\end{CJK}  %%% ZZZ %%%
\end{document} 
 